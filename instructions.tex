\title{CHESS Tomography Extension Procedure 1.0.1}
\author{
        Jakob Kaminsky \\
                Cornell University\\
}
\date{\today}
\documentclass[10pt]{article}
\usepackage[margin= 1.25in]{geometry}
\usepackage{float, graphicx}
\restylefloat{table}

\begin{document}
\maketitle

\section*{Overview}
This module is an extension for tomopy 1.0.1. It allows users to process raw tomographic data
through automation and GUIs.\\

\noindent It is designed to decrease the amount of variables necessary to complete reconstructions 
and increase ease of use, while still maintaining the flexability of tomopy.\\

\noindent By reading this manual and using the template, a new user should be able to complete a full
reconstruction with no edits to any code besides the initial inputs.
\section*{Dependencies}
\begin{table}[ht]
    % title of Table
    \centering % used for centering table
    \begin{tabular}{c c}
    \hline\hline %inserts double horizontal lines
    Software & Version\\ [0.5ex] % inserts table
    %heading
    \hline % inserts single horizontal line
    python & 2.7.14\\ % inserting body of the table
    tomopy & 1.0.1\\
    matplotlib & 2.1.0\\
    ipython & 5.4.1\\
    h5py & 2.7.0\\ [1ex] % [1ex] adds vertical space
    \hline %inserts single line
    \end{tabular}
\end{table}

\section*{Initial Data Format} There are two pieces of data that must be loaded in order for proper reconstruction: 
the tomographic image stack, and theta values for the stack. 
These should be in the format [elements, rows, angles, columns] and [radians] respectively.\\
\ \\This data should be packaged into an hdf5 file as seperate data sets. This can be done using the save() method included in this module.
\section*{Procedure}\label{Procedure}
\section{Inputs}
In order for the script to run, four parameters must be specified:
\begin{enumerate}
    \item filename: the filename of the hdf5 data \\ 
    \textit{ex: `/nfs/chess/aux/user/jk989/data.hdf5'}
    \item sinogramName: name of tomographic dataset (chosen when using save() function)\\
    \textit{ex: `tomoImgs'}
    \item thetaName: name of theta dataset (chosen when using save() function)\\
    \textit{ex: `theta'}
    \item moduleFolder: location of tomoFunctions module\\
    \textit{ex: `/nfs/chess/aux/user/jk989'}
\end{enumerate}
\section{Loading Data}
After the inputs are selected, now the module can be imported. After importation, loadHDF5() can be called.\\
\begin{center} def loadHDF5(fileName,\\sinogramName = `tomoImgs',\\ thetaName = `theta')\\
\ \\ Loads tomographic data and theta from an hdf5 file. Tomo data should be in the format [elements, rows, angles, cols]. Theta should be an array of radians.
\end{center}
\begin{table}[H]
    \centering
    \begin{tabular}{|l|l|l|}
        \hline
        \textbf{parameter} & \textbf{type} &\textbf{description}  \\ \hline
        filename & string & file to parse \\ \hline
        (optional) sinogramName & string & name of tomographic dataset \\ \hline
        (optional) thetaName & string & name of theta dataset \\ \hline
    \end{tabular}
\end{table}
{\center \textbf{Returns:} array ([tomographic data, theta])\\}

\section{Determining Bounds}
The next step is to launchBoundHelper(). This GUI allows a user to pick the bounds of interest.
The radiograph should specify the \textit{x} element and \textit{y} layer. It is recommended that
the most expansive element and closest to center layer be chosen.\\
\begin{center}
def launchBoundHelper(radiograph,\\vMin = None,\\ vMax = None,\\ cmap = None,\\ interpolation = `none')\\
\ \\Launches GUI which allows users to pick section of sample for evaluation.
\end{center}
\begin{table}[H]
    \centering
    \begin{tabular}{|l|l|l|}
        \hline
        \textbf{parameter} & \textbf{type} &\textbf{description}  \\ \hline
        radiograph & 2D ndarray (tomoImgs[\textit{x},:,\textit{y},:])& radiograph to display \\ \hline
        (optional) vmin/vmax & scalar & lower/upper bound of color spectrum \\ \hline
        (optional) cmap & Colormap & (matplotlib.colors.Colormap)\\ && function which determines color scheme \\ \hline
        (optional) interpolation & string &  determines algo for interpolation, see\\
         & & matplotlib.pyplot.imshow for list of algs \\ \hline
    \end{tabular}
\end{table}
{\center \textbf{Returns:} instance of BoundHelper\\}
\ \\ To chose the sample space, first select the top left point and then bottom right point. 
If mouse is clicked and dragged, no points will be selected. This is to ensure matplotlib's zoom functionality.
Due to the nature of the reconstruction algorithm, leave room on the left and right sides of the sample. See example:
\begin{figure}[H]
    \begin{center}
        \includegraphics[scale = .4]{/Users/jakobkaminsky/Desktop/Images/eee}
    \end{center}
\end{figure}
\noindent The $x$ coordinates determine imageBounds and the $y$ coordinates determine layerBounds.
\section{Determining Reconstruction Values}
Now the sample can be reconstructed. The center of the image must be determined in order to get a proper reconstruction.
However, because the top and bottom of the sample have different centers, we will need to find the center for the top of 
layerBounds and the center for the bottom of layerBounds. The center can be determined my changing the center slider in the GUI,
trying to optimize the clarity of the image.\\\\
Sometimes the extremes of the layerBounds do not contain the sample, so in that case, use the layer input of the GUI to change the layer.
However, keep the layer close to the top/bottom edge of the sample to ensure an accurate reconstruction.\\
\begin{center} def launchValHelper(sinograms,\\ imageBounds,\\ layer,\\ layerBounds,\\ theta,\\ sigma = .1,\\ ncore = 4,\\ algorithm = `gridrec',\\ vmin = None,\\ vmax = None,\\ cmap = `binary',\\ interpolation = `none')\\
\ \\ Launches GUI which allows users to pick reconstruction variables.
\begin{table}[H]
    \centering
    \begin{tabular}{|l|l|l|}
        \hline
        \textbf{parameter} & \textbf{type} &\textbf{description}  \\ \hline
        sinograms & 4D ndarray & [elements, rows, angles, cols] \\ \hline
        imageBounds & len 2 array & boundary of sample to be evaluated \\ \hline
        layer & int & inital layer to be displayed\\ \hline
        layerBounds&len 2 array&reconstruction layer bounds\\ \hline
        theta & ndarray & list of angles in radians\\ \hline
        (optional) sigma&float &damping param in Fourier space \\ \hline
        (optional) ncore&int & \# of cores that will be assigned \\ \hline
        (optional) algorithm& \{str, function\} & reconstruction algorithm\\
        & & see tomopy.recon.algorithm for list\\ \hline
        (optional) vmin/vmax & scalar & lower/upper bound of color spectrum \\ \hline
        (optional) cmap & Colormap & (matplotlib.colors.Colormap)\\ && function which determines color scheme \\ \hline
        (optional) interpolation & string &  determines algo for interpolation, see\\
         & & matplotlib.pyplot.imshow for list of algs \\ \hline

    \end{tabular}
\end{table}
\textbf{Returns:}  instance of ValHelper\\
\end{center}
\ \\ The launched GUI should look as follows:
\begin{figure}[H]
    \begin{center}
        \includegraphics[scale = .5]{/Users/jakobkaminsky/Desktop/Images/fff}
    \end{center}
\end{figure}
\ \\Then use getImageBounds() and getLayerBounds() to store GUI values into variables.
\section{Calculating Centers}
Now the centers for each layer must be calculated. If the top center and bottom center are the same, this step can be skipped. 
Else, use calcCenters() to calculated the centers for all layers in layerBounds.
\begin{center} 
def calcCenters(layerBounds,\\ topCenter,\\ topLayer,\\ bottomCenter,\\ bottomLayer)\\
\ \\ Takes the difference between topCenter and bottomCenter, assumes
center movement is linear, and calculates a list of centers for
layerBounds.
\begin{table}[H]
    \centering
    \begin{tabular}{|l|l|l|}
        \hline
        \textbf{parameter} & \textbf{type} &\textbf{description}  \\ \hline
        layerBounds&len 2 array&reconstruction layer bounds\\ \hline
        topLayer/bottomLayer&int&layers from top and bottom of stack w/ known centers\\ \hline
        topCenter/bottomCenter&scalar&center vals for topLayer/bottomLayer\\ \hline
    \end{tabular}
\end{table}
\textbf{Returns:} array containing extrapolated centers for layerBounds\\
\end{center}
%%%%%%%%%%%%%%%%%%%%%%%%%%%%%%%%%%%%%%%%%%
\section{Reconstruction}
Now that all parameters have been found, use the reconstruct() method to reconstruct every layer in layerBounds.
\begin{center} def reconstruct(sinograms,\\ centers,\\ imageBounds,\\ layers,\\ theta,\\ sigma = .1,\\ ncore = 4,\\ algorithm = `gridrec')\\
\ \\ Reconstructs object from projection data.
Takes in list [elements, rows, angles, cols]
or [rows, angles, cols],
and returns ndarray representing a 3D reconstruction.

\begin{table}[H]
    \centering
    \begin{tabular}{|l|c|c|}
        \hline
        \textbf{parameter} & \textbf{type} &\textbf{description}  \\ \hline
        sinograms&ndarray&3D tomographic data\\ \hline
        centers&scalar, array&estimated location(s) of rotation axis\\ \hline
        imageBounds&len 2 array&boundary of sample to be evaluated\\ \hline
        layers&scalar, len 2 array&single layer or bounds of layers\\ \hline
        theta&ndarray&list of angles in radians\\ \hline
        (optional) sigma&float&damping param in Fourier space\\ \hline
        (optional) ncore&int& \# of corse that will be assigned\\ \hline
        (optional) alogirthm&{str, function}&determines algo for interpolation, see\\
        & & matplotlib.pyplot.imshow for list of algs \\ \hline
    \end{tabular}
\end{table}
\textbf{Returns:} ndarray representing multi-elemental 3D reconstructions.\\
\end{center}
\ \\Then use getCenter() and getLayer() to store the GUI values into variables.
\section{Post-Reconstruction}
There are various tools for visualizing reconstructions. The most simple is plotLayer():
\begin{center}
    def plotLayer(image,\\ vmin = None,\\ vmax = None,\\ area = .6,\\ cmap = `binary',\\ interpolation = `none')\\
    \ \\ plots a 2D array
\begin{table}[H]
    \centering
    \begin{tabular}{|l|l|l|}
        \hline
        \textbf{parameter} & \textbf{type} &\textbf{description}  \\ \hline
        image &2D ndarray& image to display\\ \hline
        (optional) vmin/vmax & scalar & lower/upper bound of color spectrum \\ \hline
        (optional) area & (0,1] & generality of vmin/vmax estimated \\
        &&(0 includes less pixels, 1 includes all pixels including extremes)\\ \hline
        (optional) cmap & Colormap & (matplotlib.colors.Colormap)\\ && function which determines color scheme \\ \hline
        (optional) interpolation & string &  determines algo for interpolation, see\\
         & & matplotlib.pyplot.imshow for list of algs \\ \hline
    \end{tabular}
\end{table}
\textbf{Returns:} AxesImage object (plot)
\end{center}
There is also multiSliceGiffer() and multiSliceViewer(), the former cycles
through your reconstructed images automatically, and the latter cycling with
keyboard presses \{j\} and \{k\}.
\begin{center} def multiSliceGiffer(volume,\\ cmap = None)\\
\ \\ Display that cycles through slices of a volume.
\begin{table}[H]
    \centering
    \begin{tabular}{|l|l|l|}
        \hline
        \textbf{parameter} & \textbf{type} &\textbf{description}  \\ \hline
        volume&3D ndarray& volume to display\\ \hline
        (optional) cmap & Colormap & (matplotlib.colors.Colormap)\\ && function which determines color scheme \\ \hline
    \end{tabular}
\end{table}
\textbf{Returns:} AxesImage object (plot)\\
\end{center}
\ \\Documentation for multiSliceViewer() is omitted because it is the same as multiSliceGiffer except without the cmap parameter.\\

\section{Saving Data}
Once reconstructions have been finalized, save the array as and hdf5 file with save():\\
\begin{center} def save(filename,\\ names,\\ data)\\ %datatype = np.float16)\\
\ \\ Saves files as hdf5 in specified directory.
\begin{table}[H]
    \centering
    \begin{tabular}{|l|l|l|}
        \hline
        \textbf{parameter} & \textbf{type} &\textbf{description}  \\ \hline
        fileName&String& desired name of file\\ \hline
        names & [String,] & list of desired names of data\\ \hline 
        data&[ndarray,]& list of data\\ \hline
        %(optional) datatype&type&desired datatype\\ \hline
    \end{tabular}
\end{table}
\textbf{Returns:} None\\
\end{center}
\ \\ And the process is complete!
\section*{Contact}\label{Contact}
Any questions or suggestions can be sent to jk989@cornell.edu.
\end{document}